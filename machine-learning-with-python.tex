\documentclass{article}%
\usepackage[T1]{fontenc}%
\usepackage[utf8]{inputenc}%
\usepackage{lmodern}%
\usepackage{textcomp}%
\usepackage{lastpage}%
\usepackage{geometry}%
\geometry{margin=2.5cm}%
%
\title{Machine Learning With Python Notes}%
\author{Patrick Adams}%
\date{\today}%
%
\begin{document}%
\normalsize%
\maketitle%
\newpage%
Note: This is a draft copy of notes generated by free code camp.\newline%
%
https://www.freecodecamp.org/%
\newpage%
\tableofcontents%
\section{Tensorflow}%
\label{sec:Tensorflow}%
\subsection{Introduction: Machine Learning Fundamentals}%
\label{subsec:IntroductionMachineLearningFundamentals}%

%
\subsection{Introduction to TensorFlow}%
\label{subsec:IntroductiontoTensorFlow}%

%
\subsection{Core Learning Algorithms}%
\label{subsec:CoreLearningAlgorithms}%

%
\subsection{Core Learning Algorithms: Working with Data}%
\label{subsec:CoreLearningAlgorithmsWorkingwithData}%

%
\subsection{Core Learning Algorithms: Training and Testing Data}%
\label{subsec:CoreLearningAlgorithmsTrainingandTestingData}%

%
\subsection{Core Learning Algorithms: The Training Process}%
\label{subsec:CoreLearningAlgorithmsTheTrainingProcess}%

%
\subsection{Core Learning Algorithms: Classification}%
\label{subsec:CoreLearningAlgorithmsClassification}%

%
\subsection{Core Learning Algorithms: Building the Model}%
\label{subsec:CoreLearningAlgorithmsBuildingtheModel}%

%
\subsection{Core Learning Algorithms: Clustering}%
\label{subsec:CoreLearningAlgorithmsClustering}%

%
\subsection{Core Learning Algorithms: Hidden Markov Models}%
\label{subsec:CoreLearningAlgorithmsHiddenMarkovModels}%

%
\subsection{Core Learning Algorithms: Using Probabilities to make Predictions}%
\label{subsec:CoreLearningAlgorithmsUsingProbabilitiestomakePredictions}%

%
\subsection{Neural Networks with TensorFlow}%
\label{subsec:NeuralNetworkswithTensorFlow}%

%
\subsection{Neural Networks: Activation Functions}%
\label{subsec:NeuralNetworksActivationFunctions}%

%
\subsection{Neural Networks: Optimizers}%
\label{subsec:NeuralNetworksOptimizers}%

%
\subsection{Neural Networks: Creating a Model}%
\label{subsec:NeuralNetworksCreatingaModel}%

%
\subsection{Convolutional Neural Networks}%
\label{subsec:ConvolutionalNeuralNetworks}%

%
\subsection{Convolutional Neural Networks: The Convolutional Layer}%
\label{subsec:ConvolutionalNeuralNetworksTheConvolutionalLayer}%

%
\subsection{Creating a Convolutional Neural Network}%
\label{subsec:CreatingaConvolutionalNeuralNetwork}%

%
\subsection{Convolutional Neural Networks: Evaluating the Model}%
\label{subsec:ConvolutionalNeuralNetworksEvaluatingtheModel}%

%
\subsection{Convolutional Neural Networks: Picking a Pretrained Model}%
\label{subsec:ConvolutionalNeuralNetworksPickingaPretrainedModel}%

%
\subsection{Natural Language Processing With RNNs}%
\label{subsec:NaturalLanguageProcessingWithRNNs}%

%
\subsection{Natural Language Processing With RNNs: Part 2}%
\label{subsec:NaturalLanguageProcessingWithRNNsPart2}%

%
\subsection{Natural Language Processing With RNNs: Recurring Neural Networks}%
\label{subsec:NaturalLanguageProcessingWithRNNsRecurringNeuralNetworks}%

%
\subsection{Natural Language Processing With RNNs: Sentiment Analysis}%
\label{subsec:NaturalLanguageProcessingWithRNNsSentimentAnalysis}%

%
\subsection{Natural Language Processing With RNNs: Making Predictions}%
\label{subsec:NaturalLanguageProcessingWithRNNsMakingPredictions}%

%
\subsection{Natural Language Processing With RNNs: Create a Play Generator}%
\label{subsec:NaturalLanguageProcessingWithRNNsCreateaPlayGenerator}%

%
\subsection{Natural Language Processing With RNNs: Building the Model}%
\label{subsec:NaturalLanguageProcessingWithRNNsBuildingtheModel}%

%
\subsection{Natural Language Processing With RNNs: Training the Model}%
\label{subsec:NaturalLanguageProcessingWithRNNsTrainingtheModel}%

%
\subsection{Reinforcement Learning With Q{-}Learning}%
\label{subsec:ReinforcementLearningWithQ{-}Learning}%

%
\subsection{Reinforcement Learning With Q{-}Learning: Part 2}%
\label{subsec:ReinforcementLearningWithQ{-}LearningPart2}%

%
\subsection{Reinforcement Learning With Q{-}Learning: Example}%
\label{subsec:ReinforcementLearningWithQ{-}LearningExample}%

%
\subsection{Conclusion}%
\label{subsec:Conclusion}%

%
\newpage%
\section{How Neural Networks Work}%
\label{sec:HowNeuralNetworksWork}%
\subsection{How Deep Neural Networks Work}%
\label{subsec:HowDeepNeuralNetworksWork}%

%
\subsection{Recurrent Neural Networks RNN and Long Short Term Memory LSTM}%
\label{subsec:RecurrentNeuralNetworksRNNandLongShortTermMemoryLSTM}%

%
\subsection{Deep Learning Demystified}%
\label{subsec:DeepLearningDemystified}%

%
\subsection{How Convolutional Neural Networks work}%
\label{subsec:HowConvolutionalNeuralNetworkswork}%

%
\newpage%
\section{Machine Learning With Python Projects}%
\label{sec:MachineLearningWithPythonProjects}%
\subsection{Rock Paper Scissors}%
\label{subsec:RockPaperScissors}%
For this challenge, you will create a program to play Rock, Paper, Scissors. A program that picks at random will usually win 50\% of the time. To pass this challenge your program must play matches against four different bots, winning at least 60\% of the games in each match.\newline%
You can access the full project description and starter code on repl.it.\newline%
After going to that link, fork the project. Once you complete the project based on the instructions in 'README.md', submit your project link below.\newline%
We are still developing the interactive instructional part of the machine learning curriculum. For now, you will have to use other resources to learn how to pass this challenge.\newline%

%
\subsection{Cat and Dog Image Classifier}%
\label{subsec:CatandDogImageClassifier}%
For this challenge, you will use TensorFlow 2.0 and Keras to create a convolutional neural network that correctly classifies images of cats and dogs with at least 63\% accuracy.\newline%
You can access the full project instructions and starter code on Google Colaboratory.\newline%
After going to that link, create a copy of the notebook either in your own account or locally. Once you complete the project and it passes the test (included at that link), submit your project link below. If you are submitting a Google Colaboratory link, make sure to turn on link sharing for "anyone with the link."\newline%
We are still developing the interactive instructional content for the machine learning curriculum. For now, you can go through the video challenges in this certification. You may also have to seek out additional learning resources, similar to what you would do when working on a real{-}world project.\newline%

%
\subsection{Book Recommendation Engine using KNN}%
\label{subsec:BookRecommendationEngineusingKNN}%
In this challenge, you will create a book recommendation algorithm using K{-}Nearest Neighbors.\newline%
You will use the Book{-}Crossings dataset. This dataset contains 1.1 million ratings (scale of 1{-}10) of 270,000 books by 90,000 users.\newline%
You can access the full project instructions and starter code on Google Colaboratory.\newline%
After going to that link, create a copy of the notebook either in your own account or locally. Once you complete the project and it passes the test (included at that link), submit your project link below. If you are submitting a Google Colaboratory link, make sure to turn on link sharing for "anyone with the link."\newline%
We are still developing the interactive instructional content for the machine learning curriculum. For now, you can go through the video challenges in this certification. You may also have to seek out additional learning resources, similar to what you would do when working on a real{-}world project.\newline%

%
\subsection{Linear Regression Health Costs Calculator}%
\label{subsec:LinearRegressionHealthCostsCalculator}%
In this challenge, you will predict healthcare costs using a regression algorithm.\newline%
You are given a dataset that contains information about different people including their healthcare costs. Use the data to predict healthcare costs based on new data.\newline%
You can access the full project instructions and starter code on Google Colaboratory.\newline%
After going to that link, create a copy of the notebook either in your own account or locally. Once you complete the project and it passes the test (included at that link), submit your project link below. If you are submitting a Google Colaboratory link, make sure to turn on link sharing for "anyone with the link."\newline%
We are still developing the interactive instructional content for the machine learning curriculum. For now, you can go through the video challenges in this certification. You may also have to seek out additional learning resources, similar to what you would do when working on a real{-}world project.\newline%

%
\subsection{Neural Network SMS Text Classifier}%
\label{subsec:NeuralNetworkSMSTextClassifier}%
In this challenge, you need to create a machine learning model that will classify SMS messages as either "ham" or "spam". A "ham" message is a normal message sent by a friend. A "spam" message is an advertisement or a message sent by a company.\newline%
You can access the full project instructions and starter code on Google Colaboratory.\newline%
After going to that link, create a copy of the notebook either in your own account or locally. Once you complete the project and it passes the test (included at that link), submit your project link below. If you are submitting a Google Colaboratory link, make sure to turn on link sharing for "anyone with the link."\newline%
We are still developing the interactive instructional content for the machine learning curriculum. For now, you can go through the video challenges in this certification. You may also have to seek out additional learning resources, similar to what you would do when working on a real{-}world project.\newline%

%
\newpage%
\end{document}